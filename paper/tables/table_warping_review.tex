\begin{table}[htbp]
  \centering
  \caption{Summary and assessment the warping methods for trajectory synchronization}
    \resizebox{\textwidth}{!}{
    \footnotesize
    \begin{tabular}{l p{4cm} p{2cm} p{2cm} p{2cm}}
    \toprule
    Method & Description & Advantages & Disadvantages & Speed \\
    \midrule
    \parbox[t]{2cm}{Truncation and Padding} & Delete (longer) the extra time series or pad (shorter) time series data to the same length based on a golden reference trajectory & Easy-to-implement & Does not explicitly align features & Fast \\
    \parbox[t]{2cm}{Linear Time Scaling} & Use a monotonic batch evolution indicator variable (percent completion, product concentration, tank level) and stretch or shrink linearly to ensure every trajectory is of same length & Straightforward & Performance dependent on good indicator variable & Fast \\
    \parbox[t]{2cm}{Dynamic Time Warping} & Solves dynamic programming problem to find the most similar path between two trajectories & Explicit feature alignment & Information loss, trajectory distortion, implementation challenge & Medium \\
    \parbox[t]{2cm}{Correlation Optimized Warping} & Solves dynamic programming problem with sub-optimization routines to maximize correlation between trjaectory & Explicit feature alignment, less distortion than DTW & Information loss, complex, implementation challenge & Slow \\
    \bottomrule
    \end{tabular}}
  \label{tab:warping_review}%

\end{table}%
